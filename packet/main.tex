\documentclass[12 pt]{article}
\hbadness=10


\usepackage{amsmath, amssymb, amsfonts, setspace, stmaryrd, amsthm, graphicx, tikz}

\usepackage[utf8]{inputenc}
\usepackage[english]{babel}
\usetikzlibrary{positioning}% To get more advanced positioning options
\usetikzlibrary{arrows}% To get more arrow heads
\usepackage{hyperref} % Allows to make references and links clickable

\usepackage[margin=1.5in]{geometry}
\newtheorem{theorem}{Theorem}
\newtheorem{definition}{Definition}
%%align* environment
\newcommand{\eq}[1]{\begin{align*}#1\end{align*}}

%Math commands
\DeclareMathOperator{\R}{\mathbb{R}}
\DeclareMathOperator{\C}{\mathbb{C}}
\DeclareMathOperator{\Z}{\mathbb{Z}}
\DeclareMathOperator{\N}{\mathbb{N}}
\DeclareMathOperator{\Q}{\mathbb{Q}}

%Crypto commands
\usepackage{cryptocode}
\newcommand{\sample}{\hskip2.3pt{\gets\!\!\mbox{\tiny${\$}$\normalsize}}\,}
\def\D{\ensuremath{\mathcal{D}}}
\def\A{\ensuremath{\mathcal{A}}}
\def\ss{\ensuremath{\mathcal{S}}}
\def\bin{\ensuremath{\{0, 1\}}}

%comments
\newcommand{\nsm}[1]{\textcolor{teal}{(NG: #1)}}
\newcommand{\todo}[1]{\textcolor{red}{#1}}

\renewcommand\abstractname{Introduction}

\newcounter{exercise}[section]
\newenvironment{exercise}{\refstepcounter{exercise}\par\bigskip \begin{quotation}{}{\leftmargin .25in\rightmargin .25in}
    \noindent \textbf{Exercise~\thesection.\theexercise }  \rmfamily}{\end{quotation}\par\bigskip}

\newenvironment{bonus}{\refstepcounter{exercise}\par\bigskip \begin{quotation}{}{\leftmargin .25in\rightmargin .25in}
    \noindent \textbf{*Bonus Exercise*~\thesection.\theexercise }  \rmfamily}{\end{quotation}\par\bigskip}

\newcounter{example}[section]
\newenvironment{example}{\refstepcounter{example}\par\bigskip \begin{quotation}{}{\leftmargin .25in\rightmargin .25in}
    \noindent \textbf{Example~\thesection.\theexercise }  \rmfamily}{\end{quotation}\par\bigskip}

\title{Cryptographic Secret Sharing}
\author{Girls Talk Math}
\date{}

\begin{document}
\maketitle
\vskip 1in
\begin{center} \textbf{Introduction} \end{center}

\indent In this problem set, you will learn about ...
    
    One last note about reading mathematical texts: it is very normal when reading math to read a passage or even a single sentence several times before understanding it properly. Also, never trust the author! Check every claim and calculation (time permitting). Take your time and never give up. Let's talk math!
	
\newpage

\tableofcontents

%%%%%%%%%%%%%%%%%%%%%%
%%% Use \section{}, \subsection{}, etc. for different parts of the problem set. Start a newpage for a new section
%%%%%%%%%%%%%%%%%%%%%%

%%%%%%%%%%%%%%%%%%%%%%%%%%%%%%%%%
%%%
%%% Embed exercises in section as appropriate 
%%% 
%%%%%%%%%%%%%%%%%%%%%%%%%%%%%%%%%%%

\newpage


\section{Probability and Randomness}

\subsection{Introduction}
\nsm{Define probability, define what it means to be 
random. Introduce basic notation.}

\subsection{Randomness in Cryptography}
\nsm{Define \emph{information-theoretic security}.
Define what a security parameter is ($\lambda$).}

\subsection{Sharing Secrets}
\emph{Secret sharing} is a way to ``split'' a secret value (call it $x$) into pieces, called \emph{shares}. 
Two useful properties a secret-sharing might have are \emph{correctness} and 
\emph{privacy}. Informally, correctness means that if we put shares back together, 
we get back the original secret; privacy says that each share by itself reveals 
nothing about the secret $x$. 

These properties are important for practical uses of secret sharing, for 
example distributing shares among a large set of people so that no one 
knows the secret but some subset of them can recover the secret if they 
pool their information.

\begin{example}
    \nsm{Simple additive secret-sharing.}
\end{example}

Let's be a little more rigorous about these definitions, using the standard
notation in cryptography. First, we'll define what a secret sharing scheme 
does without giving implementation details (there could be multiple ways 
of achieving the same thing, after all).

\def\share{\ensuremath{\mathsf{Share}}}
\def\rec{\ensuremath{\mathsf{Rec}}}
\begin{definition}[Secret sharing scheme]
    Let $\D$ be the input domain and $\D_s$ be the share domain.
    A secret sharing scheme is a pair of efficient algorithms $(\share, \rec)$
    such that

    \begin{itemize}
        \item \share~takes as input a secret $x \in \D$ and two numbers $t, n$ 
        and outputs $n$ shares in $\D_s$.
        \item \rec~takes as input $m$ shares $s_1, \ldots, s_m \in D_s$. 
        If $m < t$, it outputs a special symbol $\perp$ indicating failure; 
        otherwise, it outputs some value $y \in \D$.
    \end{itemize}

    The scheme is \emph{correct} if for all $x \in \D$, $\rec(\share(x)) = x$.
\end{definition}

Here's a visual representation:

\nsm{Picture of two domains, with \share~ mapping in one direction and 
\rec~ back in the other.}

% \begin{definition}[secret-sharing scheme]
%     A pair of polynomial-time algorithms $\Sigma = (\share, \recon)$ is a (two-party) secret-sharing scheme if
%     \begin{itemize}
%         \item $\share$ takes as input a security parameter $\secparam$ and a value $x$ in the domain $\mathcal{D}_\kappa$ associated with~$\kappa$ (e.g., $\mathcal{D}_\kappa = \bin^\kappa$) and outputs two shares $\sh_1, \sh_2$. We assume $\kappa$ is implicit in each share.
%         \item $\recon$ takes as input two shares and outputs either a value $y \in \mathcal{D}_\kappa$ or a distinguished symbol $\perp$.
%     \end{itemize}
%     For correctness, we require that for all $\kappa$ and $x \in \mathcal{D}_\kappa$, $\recon(\share(\secparam, x)) = x$.
% \end{definition}

\nsm{Define the privacy property and the cryptographic privacy game? The goal can be to have a (bonus) exercise in which they write a real cryptographic proof of privacy of a secret sharing scheme (likely the XOR scheme).}

\begin{example}
    \nsm{Secret sharing without correctness.}
\end{example}

\begin{exercise}
    \nsm{Is this scheme private?}
\end{exercise}

In the next sections, we'll see some examples of private secret sharing schemes.

\newpage
\section{A simple secret sharing}

\subsection{Binary Arithmetic}

The exclusive-OR (XOR) operation is denoted by the symbol $\oplus$ and defined by the following truth table:

\begin{center}
\begin{tabular}{c|c|c}
    $a$ & $b$ & $a \oplus b$\\\hline
    0 & 0 & 0\\
    0 & 1 & 1\\
    1 & 0 & 1\\
    1 & 1 & 0
\end{tabular}
\end{center}

\nsm{explain binary representations of numbers and binary arithmetic.}

\begin{exercise}
    Write the following numbers in binary.
    \begin{enumerate}\renewcommand{\labelenumi}{\arabic{enumi})}
        \item 5
        \item 12
        \item 15
        \item 16
        \item 23
    \end{enumerate}
\end{exercise}

\begin{exercise}
    Convert the following numbers to base 10.
    \begin{enumerate}\renewcommand{\labelenumi}{\arabic{enumi})}
        \item 11
        \item 110
        \item 1001
        \item 11110
        \item 110101
    \end{enumerate}
\end{exercise}

\begin{exercise}
    \begin{enumerate}\renewcommand{\labelenumi}{\arabic{enumi})}
        \item $101 \oplus 110$
        \item $1 \oplus 111$
        \item $1101001 \oplus 0010110$
        \item $1101001 \oplus 1101001$
    \end{enumerate}
\end{exercise}

\begin{exercise}
    \begin{enumerate}\renewcommand{\labelenumi}{\arabic{enumi})}
        \item $4 \oplus 5$
        \item $11 \oplus 1$
        \item $8 \oplus 4$
        \item $7 \oplus 3$
    \end{enumerate}
\end{exercise}

\subsection{Sharing Secrets using XOR}

\begin{align*}
    & \underline{\textsf{Share}(s)} \\
    & s_1 \sample \{1, \ldots, 2^\lambda\} \\
    & \textsf{return}~ (s_1, s \oplus s_2)
\end{align*}

\begin{align*}
    & \underline{\textsf{Reconstruct}(s_1, s_2)} \\
    & \textsf{return}~ s_1 \oplus s_2
\end{align*}

\begin{exercise}
    Pick your favorite number and use the \share~algorithm defined above 
    to secret share it.
\end{exercise}

\begin{exercise}
    What is $\rec(14, 21, 49)$?
\end{exercise}

\subsubsection{Proving Security*}
\nsm{Optional section. Introduce the privacy game, have them act as the adversary and try to beat the privacy and get some intuition about why it's hard, then work through a simple proof.}

In cryptography, security properties like privacy are defined using what 
are called ``games''. A game is a challenge in which an attacker (called 
the \emph{adversary} and usually denoted by a curly $\mathcal{A}$) is given 
some information and has the goal of providing an answer that somehow 
breaks the security property of a scheme. 

For example, in the case of privacy for a secret sharing scheme, we give 
the attacker a share and ask it to give us some information about the 
secret it came from. This should be almost impossible if the scheme is 
private.

Here is the \emph{privacy game} for any secret sharing scheme $\ss = (\share, \rec)$:
\begin{enumerate}
    \item The adversary \A~picks two values $x_0, x_1$ and a number $i$ 
    between 1 and $n$.
    \item The game flips a coin to randomly choose one of those two values.
    This is usually written as picking a random value $b$ from the set 
    $\{0, 1\}$.
    \item Now the game runs the \share~algorithm on this randomly chosen 
    value $x_b$ to get shares $s_1, \ldots, s_m$. It gives the $i$th share
    to \A.
    \item \A~tries to guess which of the two values $x_0, x_1$ were shared.
    More specifically, it outputs a guess $b' \in \{0, 1\}$. 
    \item If $b=b'$, we say the result of the game is 1 (to signify 
    \textbf{true} or \textbf{success}), in which case we say that \A~``wins''
    the game; otherwise, it's 0.
\end{enumerate}

In cryptography, these games are generally written much more compactly by 
using symbols. The privacy game above would be written as follows (without 
the comments, which are there to explain the notation):

\bigskip
\begin{center}\fbox{%
\pseudocode[syntaxhighlight=auto,head=SS-priv$_{\A,\ss}$]{% (t,n)
    (x_0, x_1, i) \leftarrow \A
    ~\pccomment{get input values from $\A$} \\
    b \sample \bin 
    ~\pccomment{pick value to share at random} \\
    s_1, \ldots, s_m \gets \share(x_b, t, n) 
    ~\pccomment{share the chosen value} \\
    b' \gets \A(s_i) 
    ~\pccomment{$\A$ uses the $i$th share to guess $b'$} \\
    return b=b' 
    ~\pccomment{return 1 if $b=b'$ and 0 otherwise}
}
}\end{center}

% \begin{center}\begin{minipage}{0.5\textwidth}
%     \begin{algorithmic}
%         \STATE $(x_0,x_1, i) \leftarrow \A(\secparam)$
%         \STATE $b \sample \bin$
%         \STATE $(\sh_{b,1}, \sh_{b,2}) \leftarrow \share(\secparam, x_b)$
%         \STATE $b' \leftarrow \A(\sh_{b,i})$ 
%         \RETURN $b = b'$
%     \end{algorithmic}\end{minipage}\end{center}

\begin{exercise}
    \todo{With a partner, play through the security a couple times using the 
    XOR secret sharing scheme. One of you should take on the role of the 
    game while the other acts as the adversary. After you've done this a 
    couple of times, switch roles and repeat.}

    How successful was the adversary? If you were the adversary, what was 
    difficult about your role? What was the key part of the scheme that 
    ensured privacy?

    \nsm{Maybe try again with another scheme that is not deterministic but 
    gives the adversary >1/2 odds of winning?}
\end{exercise}

Now that we have an idea of why the game is difficult to win consistently, 
let's rigorously define privacy by specifying how often the attacker should 
be able to win: 

\begin{definition}[secret sharing privacy]
    A secret sharing scheme $\ss = (\share,\allowbreak \rec)$ is \emph{private} if,
    for all realistic\footnotemark adversaries \A,
    \footnotetext{In cryptography we usually consider what are called 
    \emph{probabilistic polynomial-time}, or PPT, adversaries.}

    \[
        \Pr[\textnormal{SS-priv}_{\A,\ss}(t,n) = 1] - \frac{1}{2}
    \]

    is small\footnotemark.
    \footnotetext{\todo{bounded by a \emph{negligible} function.}}
\end{definition}

What does this actually mean? No matter what adversary \A~we are dealing with, 
\todo{the probability that it can win the privacy game should be very close
to one-half}, which is what it should be if \A~were to randomly guess which 
value was shared.


\newpage
\section{Shamir's Secret Sharing}

\subsection{Polynomials}
\nsm{y-intercept, zeroes}

\subsubsection{Uniqueness}
\nsm{How many points uniquely define a polynomial}

\subsubsection{Lagrange Interpolation}

\begin{exercise}
    \nsm{Practice some manual Lagrange interpolation. Pick a polynomial, evaluate it at 3 points, then use those 3 points in Lagrance Interpolation and recover the polynomial.}
\end{exercise}

\subsection{Sharing Secrets Using Polynomials}

Shamir secret sharing is a $(t+1)$-out-of-$n$ secret sharing protocol, for some numbers $t$ and $n$. This means that we split the secret $s$ into $n$ values and distribute them to $n$ people. Then, at least $t+1$ of those people must work together to recover $s$.

\nsm{Don't introduce finite fields but maybe make a note that this should be done over finite fields.}

\nsm{Use the Jupyter Notebook to play around with this.}

\end{document}
