\section{Secret Sharing}\label{sec:ss}
\emph{Secret sharing} is a way to ``split'' a secret value (call it $s$ for 
``secret'') into pieces, called \emph{shares}. 
Two useful properties a secret sharing scheme might have are \emph{correctness} and 
\emph{privacy}. Informally, correctness means that if we put shares back together, 
we get back the original secret; privacy says that each share by itself reveals 
nothing about the secret $s$. 

These properties are important for practical uses of secret sharing. For 
example, using secret sharing, we can distribute shares among a large set of people so that no one 
knows the secret but some subset of them can recover the secret if they 
pool their information.

We'll be a little more rigorous about these definitions soon, but first, let's 
see an example.  

\subsection{A simple secret sharing}

Here is a simple scheme for sharing integers using nothing but addition and subtraction:

\def\share{\ensuremath{\mathsf{Share}}}
\def\rec{\ensuremath{\mathsf{Rec}}}

\begin{figure}[h!]
\begin{pchstack}[center]
\fbox{%
\procedure{$\share(s)$}{%
    s_1 \sample \{1, \ldots, 2^\lambda\} \\
    \pcreturn (s_1, s - s_1)
}
\pchspace
\procedure{$\rec(s_1, s_2)$}{%
    \pcreturn s_1 + s_2
}
}
\end{pchstack}
\caption{Additive secret sharing scheme}
\label{fig:additiveSS}
\end{figure}

Let's go through the notation together. First, on the left, we see that 
the $\share$ algorithm is being defined. (An \emph{algorithm} is simply a procedure.)
The parentheses after the name tell us that it takes an input $s$, which in 
this context is the secret to be shared. The first line tells us to sample\footnotemark~
an element from the set $\{1, \ldots, 2^\lambda\}$ and call it $s_1$. The 
dots in the set are shorthand for all the numbers in between. For instance, 
when $\lambda=10$, we sample $s_1$ uniformly from the set $\{1, 2, 3, 4, \ldots, 
1022,1023, 1024\}$. Now we're almost done! The next line says to return 
(output) a pair of numbers: the random number $s_1$ and the difference $s-s_1$.
\footnotetext{We introduced the symbol $\sample$ in Section~\ref{sec:prob}.}

To summarize, $\share$ converts a secret $s$ into a two shares. Okay, so how can 
two people, each with one of the shares, get back the original secret? That's
what the right side of Figure~\ref{fig:additiveSS} tells us. The reconstruction 
algorithm $\rec$ takes two integers (call them $s_1$ and $s_2$) and returns their 
sum. That's it!

If $\rec$ receives two shares that were produced by the $\share$ algorithm, 
it will return $s_1 + (s-s_1) = s$. This means the scheme has correctness!
What about privacy? 

It turns out the scheme is private as well: someone who sees only one of the two 
shares learns nothing about the secret. This is pretty straightforward if the 
one share you see is $s_1$: we picked this value randomly (remember this means 
we picked it independently of $s$), so it has nothing to do with $s$. This 
is the case with the share $s-s_1$ as well. If we're given a number $s_2$ 
calculated as $s-s_1$, we don't know the other share $s_1$, so $s_2$ could 
be anything: $s-0$, $s-1$, $s-2$, and so on. Another way to think about this 
is that we don't know $s_1$, so we can't undo the subtraction and recover $s$.
In this case we say that $s_1$ ``masks'' $s$.

These arguments for correctness and privacy are not very rigorous. We'll 
take a look at how cryptographers prove these properties in Sections~\ref{sec:formal-defs} 
and \ref{sec:proof}.

\begin{exercise}
    Pick your favorite number and secret share it using the \share~algorithm 
    defined above, with $\lambda=10$. (Hint: $2^{10}=1024$.)
    (Repeat this exercise until you are comfortable with this secret sharing 
    scheme.)
\end{exercise}

\begin{exercise}
    What is
    \renewcommand{\labelenumi}{(\alph{enumi})} 
    \begin{enumerate}%[label={(\arabic*)}] %\usepackage{enumitem}
        \item $\rec(2, 6)$?
        \item $\rec(4, 1)$?
        \item $\rec(10, 2)$?
        \item $\rec(115, -103)$?
        \item $\rec(559, -544)$?
    \end{enumerate}
\end{exercise}

\begin{exercise}
    Can you adapt this additive secret sharing scheme to output 
    3 shares instead of 2? How does this change the reconstruction 
    algorithm?
\end{exercise}

\subsection{Formal Definitions*}\label{sec:formal-defs}

We'll now formally define what it means to be a secret sharing scheme and the 
properties such a scheme might have, using the standard notation in 
cryptography. First, we'll define what a secret sharing scheme 
does without giving implementation details (there could be multiple ways 
of achieving the same thing, after all). 

% This is the n-out-of-n definition
\begin{definition}[Secret sharing scheme]\label{def:ss}
    Let $\D$ be the input domain and $\D_s$ be the share domain.
    A secret sharing scheme is a pair of efficient algorithms $(\share, \rec)$
    and an associated natural number $n$ such that

    \begin{itemize}
        \item \share~takes as input a secret $s \in \D$ and outputs $n$ 
        shares in $\D_s$.
        \item \rec~takes as input $m$ shares $s_1, \ldots, s_m \in D_s$ 
        and outputs some value $y \in \D$ or a special symbol $\perp$ 
        indicating failure. (If $m \neq n$, it outputs $\perp$.)
    \end{itemize}
\end{definition}
% from Input Auth paper
% \begin{definition}[secret-sharing scheme]
%     A pair of polynomial-time algorithms $\Sigma = (\share, \recon)$ is a (two-party) secret-sharing scheme if
%     \begin{itemize}
%         \item $\share$ takes as input a security parameter $\secparam$ and a value $x$ in the domain $\mathcal{D}_\kappa$ associated with~$\kappa$ (e.g., $\mathcal{D}_\kappa = \bin^\kappa$) and outputs two shares $\sh_1, \sh_2$. We assume $\kappa$ is implicit in each share.
%         \item $\recon$ takes as input two shares and outputs either a value $y \in \mathcal{D}_\kappa$ or a distinguished symbol $\perp$.
%     \end{itemize}
%     For correctness, we require that for all $\kappa$ and $x \in \mathcal{D}_\kappa$, $\recon(\share(\secparam, x)) = x$.
% \end{definition}


Here's a visual representation: 

\begin{center}
\pgfdeclarelayer{background}
\pgfsetlayers{background,main}
\begin{tikzpicture}[
    ovalnode/.style={ellipse, draw, fill=gray!10,align=center,minimum width=50pt, minimum height=100pt, thick},
]
    % Nodes
    \node[ovalnode] (domain)                         {$\D$};
    \node[ovalnode] (range1)  [right=of domain]      {$\D_s$};
    \begin{pgfonlayer}{background}
        \node[ovalnode] [right=of range1,xshift=-0.35cm] {$\D_s$};
        \node[text width=3cm]     at (5.75,1.6)              {$\cdots$};
        \node[text width=3cm]     at (5.75,-1.6)              {$\cdots$};
        \node[ovalnode] [right=of range1,xshift=-1.8cm] {};
        \node[ovalnode] [right=of range1,xshift=-2.3cm] {};
    \end{pgfonlayer}
    \draw [decorate,decoration={brace,amplitude=10pt,mirror},yshift=0pt]
    (2.2,-2) -- (6,-2) node [black,midway,yshift=-1.5em] 
    {\footnotesize $n$ copies};

    % \node[text width=3cm]     at (5.45,0)              {$\cdots$};
    % \node[ovalnode] (rangen)  [right=of range1]      {$\D_s$};

    % Lines
    \draw[thick,->] (domain.north east) to[out=30,in=150] 
    node[draw=none,fill=none,font=\small,midway,above] {$\share$}
    (range1.north west);
    \draw[thick,->] (range1.south west) to[out=-150,in=-30]
    node[draw=none,fill=none,font=\small,midway,below] {$\rec$}
    (domain.south east);
\end{tikzpicture}
\end{center}

\subsubsection{Correctness}

\begin{definition}[Correctness]
    A secret sharing scheme is \emph{correct} if for all $s \in \D$, $\rec(\share(s)) = s$.
\end{definition}

\begin{example}
    Consider the following secret sharing scheme:
    \begin{pchstack}[center]
    % \fbox{%
    \procedure{$\share(s)$}{%
        s_1 \sample \{1, \ldots, 2^\lambda\} \\
        \pcreturn (s_1, s + s_1)
    }
    \pchspace
    \procedure{$\rec(s_1, s_2)$}{%
        \pcreturn s_1 + s_2
    }
    % }
    \end{pchstack}
    This scheme meets Definition~\ref{def:ss} with $n=2$. The input and share domains
    are both the integers, \share~correctly outputs two integers, and 
    \rec~takes two integers and outputs another integer. \new{(If we give $\rec$
    any other number of integers, it returns $\perp$; for simplicity, we'll 
    leave this detail out of the algorithms in this section.)}

    Does this scheme have correctness? Answer in your head before reading on.

    If we compare this to the simple scheme in Figure~\ref{fig:additiveSS},
    the answer is pretty clear: no, this does not meet the correctness 
    requirement. A little bit of arithmetic confirms this: $\rec(\share(s))
    = \rec(s_1, s+s_1) = s_1 + (s+s_1) = s + 2s_1$, which is not equal to $s$
    ($s_1$ is never 0 since it is always greater than or equal to $1$). 
\end{example}

\begin{example}\label{ex:floor3}
    Let's look at another scheme:
    \begin{pchstack}[center]
    \procedure{$\share(s)$}{%
        \pcreturn (3, \lfloor s/3 \rfloor)
    }
    \pchspace
    \procedure{$\rec(s_1, s_2)$}{%
        \pcreturn s_1 \cdot s_2
    }
    \end{pchstack}
    The $\lfloor \cdot \rfloor$ notation is the ``floor'' operation, which 
    rounds a decimal number down to the integer below it. For example, $\lfloor 3.725 
    \rfloor = 3$.

    Again, this is a secret sharing scheme according to Definition~\ref{def:ss}
    with both $\D$ and $\D_s$ being the set of integers.
    But is it correct? Answer to yourself again before reading on.

    This example is a little trickier because it works in some cases -- but not all! 
    When $s=15$, 
    \[
        \rec(\share(15)) = \rec(3,5) = 3\cdot 5 = 15,
    \]
    which is correct.
    But in the case of $s=14$, 
    \[
        \rec(\share(14)) = \rec(3, 4) = 3 \cdot 4 = 12,
    \]
    which is not equal to 14.
    So this scheme isn't always correct, and since the definition of correctness 
    is all or nothing\footnotemark, the scheme doesn't have correctness.
    \footnotetext{we said $\rec(\share(s)) = s$ must hold for \emph{all} 
    $s \in \D$. But 14 was an integer for which the equality didn't hold.}
\end{example}

\begin{example}\label{ex:floor}
    So always setting $s_1$ to 3 works when sharing some numbers but not 
    others. Let's go back to picking $s_1$ randomly:
    \begin{pchstack}[center]
    % \fbox{%
    \procedure{$\share(s)$}{%
        s_1 \sample \{1, \ldots, 2^\lambda\} \\
        \pcreturn (s_1, \lfloor s/s_1 \rfloor)
    }
    \pchspace
    \procedure{$\rec(s_1, s_2)$}{%
        \pcreturn s_1 \cdot s_2
    }
    % }
    \end{pchstack}
    This still meets Definition~\ref{def:ss} with the same input and share 
    domains. Is the adjusted scheme correct now?

    What happens if we share the number 14? First, we pick $s_1$ randomly.
    Say we happen to choose 7. Then 
    \[
        \rec(\share(14)) = \rec(7, 2) = 14,
    \]
    which is correct. But what if we had randomly chosen $s_1$ to be 
    5? Then 
    \[
        \rec(\share(14)) = \rec(5, 2) = 10,
    \]
    which is not 14!

    When $s=37$, a similar thing happens: the scheme is correct for some 
    choices of $s_1$, but not others:
    \begin{align*}
        s_1 &= 1: &\rec(\share(37)) &= \rec(1, 37) = 37 = 37~\text{\cmark}\\
        s_1 &= 2: &\rec(\share(37)) &= \rec(2, 18) = 36 \neq 37~\text{\xmark}\\
        s_1 &= 3: &\rec(\share(37)) &= \rec(3, 12) = 36 \neq 37~\text{\xmark}\\
        &\vdots   &\vdots &
    \end{align*}
    So even though this scheme \emph{can} work for all $s \in \D$, correctness 
    will only hold for certain choices of $s_1$, and we have no control over 
    $s_1$ because it is chosen randomly. So this scheme is still not correct. 
\end{example}

Even though all the examples above didn't meet correctness, remember that
correct secret sharing schemes do exist. For example, we already argued informally that 
the simple additive scheme from Figure~\ref{fig:additiveSS} meets correctness
by observing that $\rec(\share(s)) = \rec(s_1, s-s_1) = s_1 + 
(s-s_1) = s + (s_1-s_1) = s$.

Now that we understand correctness, let's look at privacy.

\subsubsection{Privacy}

In cryptography, security properties like privacy are defined using what 
are called ``games''. A game is a challenge in which an attacker (called 
the \emph{adversary} and usually denoted by the curly letter $\mathcal{A}$) is given 
some information and tries to break the security property of the scheme. 
$\A$ ``wins'' the game if it can give an answer that proves it broke 
the security property of the scheme. 

For example, in the case of privacy for a secret sharing scheme, we give 
the attacker a share and ask it to give us some information about the 
secret it came from. This should be almost impossible if the scheme is 
private.

More specifically, the \emph{secret sharing privacy game} captures 
the following scenario. Alice wants to share some secret $s$ so that she
can distribute its pieces to her friends. Maybe $s$ is the combination to
her safe holding all her savings, and if something happens to her she wants
her friends to be able to access them and distribute them according to
her will. Unbeknownst to her, her friend Eve wants to go against Alice's
wishes and get into the safe herself to steal Alice's money.

Let's think about the worst case, in which Eve is very powerful: she knows 
that the safe is locked with equal probability by one of two combinations
$x_0$ or $x_1$; even worse, she was able to choose $x_0$ and $x_1$ by 
somehow influencing Alice as she chose her combination; and she can control
where in the list of friends she is so that she can pick whether she gets
share number 1, 2, \ldots, $n-1$, or $n$. Once Alice has shared her
combination among her friends, Eve will see the share $s_i$ (for her choice
of $i$), which is the result of either $\share(x_0)$ or $\share(x_1)$.

What sort of protection does Alice need? Even with all the information 
known to Eve (her knowledge of $x_0, x_1, i$, and $s_i$), we don't want her to figure
out whether she got a share of $x_0$ or $x_1$, because if she did, she'd
be able to unlock the safe!

Here is the formal privacy game for any secret sharing scheme $\ss =
(\share,\allowbreak \rec)$. (You can think of $\A$ as Eve.)
\begin{enumerate}
    \item The adversary \A~picks two values $x_0, x_1$ and a number $i$ 
    between 1 and $n$.
    \item The game flips a coin to randomly choose one of those two values.
    This is usually written as picking a random value $b$ from the set 
    $\{0, 1\}$.
    \item Now the game runs the \share~algorithm on this randomly chosen 
    value $x_b$ to get shares $s_1, \ldots, s_m$. It gives the $i$th share
    to \A.
    \item \A~tries to guess which of the two values $x_0, x_1$ were shared.
    More specifically, it outputs a guess $b' \in \{0, 1\}$. 
    \item If $b=b'$, we say the result of the game is 1 (to signify 
    \textbf{true} or \textbf{success}), in which case we say that \A~``wins''
    the game; otherwise, it's 0.
\end{enumerate}

In cryptography, these games are generally written much more compactly by 
using symbols. The privacy game above would be written as follows (without 
the comments, which are there to explain the notation):

\begin{figure}[h!]
\begin{center}\fbox{%
\pseudocode[syntaxhighlight=auto,head=SS-priv$_{\A,\ss}$]{% (t,n)
    (x_0, x_1, i) \leftarrow \A
    ~\pccomment{get input values from $\A$} \\
    b \sample \bin 
    ~\pccomment{decide randomly which value to share} \\
    s_1, \ldots, s_m \gets \share(x_b) 
    ~\pccomment{share the chosen value} \\
    b' \gets \A(s_i) 
    ~\pccomment{$\A$ uses the $i$th share to guess $b'$} \\
    return b=b' 
    ~\pccomment{return 1 if $b=b'$ and 0 otherwise}
}
}\end{center}
\caption{The secret sharing privacy game.}
\label{fig:ss-priv}
\end{figure}

Where did this definition of the game come from, you might ask? The simple 
answer is that usually the person who invents a new cryptographic 
primitive (what we call a general idea like secret sharing or 
encryption, as opposed to a specific algorithm for actually achieving 
what it describes) also gives definitions for its potential properties. 
This includes defining security games. (Sometimes, others come along 
later and describe new properties for an existing primitive; in this 
case, they might describe a game for the new property.) 

If you look at many security games, though, you'll see that they are 
usually pretty similar to each other. This is because it's useful for a 
new game to be easy to work with, since it makes other people more likely 
to build on that work. If a new game is similar to an already existing game, 
people who are familiar with the previous game can understand the new game 
quickly and prove things about a new scheme more easily.

\begin{bonus}
    With a partner, play through the privacy game a couple times using the 
    additive secret sharing scheme. One of you should take on the role of the 
    game while the other acts as the adversary. If you're playing the part 
    of the game, make sure you use something truly random, like a coin 
    flip or the Python command \texttt{random.randint(0,1)}\footnotemark,
    to pick your bit $b$.\footnotetext{Be sure to \texttt{import random} first.}
    After you've done this a couple of times, switch roles and repeat.

    How successful was the adversary? If you were the adversary, what was 
    difficult about your role? What was the key part of the scheme that 
    ensured privacy?
\end{bonus}

Now that we have an idea of why the game is difficult to win consistently, 
let's rigorously define privacy by specifying how often the adversary should 
be able to win: 

\begin{definition}[secret sharing privacy]
    A secret sharing scheme $\ss = (\share,\allowbreak \rec)$ is \emph{private} if,
    for all 
    % realistic\footnotemark % this is info-theoretic security, so we can allow unbounded adversaries
    adversaries \A,
    % \footnotetext{In cryptography we usually consider what are called 
    % \emph{probabilistic polynomial-time}, or PPT, adversaries.}

    \[
        \Pr[\textnormal{SS-priv}_{\A,\ss}(n) = 1] - \frac{1}{2}
    \]

    is small\footnotemark. This quantity is called $\A$'s \emph{advantage}.
    \footnotetext{In cryptography, this usually means bounded by a \emph{negligible} 
    function. For the purposes of this packet, ``small'' means very close to 0, 
    for example $\frac{1}{2^\lambda}$, which for $\lambda=10$ is $\frac{1}{1024}$. 
    Since the advantage ranges from 0 to $\frac{1}{2}$, a value like $\frac{1}{4}$ 
    or $\frac{1}{8}$ is not small.}
\end{definition}

What does this actually mean? If $\A$'s advantage is small, it means that 
$\Pr[\textnormal{SS-priv}_{\A,\ss}(n)=1]$ is very close to one-half. In 
other words, no matter what adversary \A~we are dealing with, the probability 
that it can win privacy game (i.e., the game outputs 1) should be very close
to one-half, which is what it should be if \A~were to randomly guess which 
value was shared.

This might sound very different from the informal definition from Section~\ref{sec:ss}:
each share by itself reveals nothing about the secret. The game instead says 
that $\A$ can't tell the difference between two different secrets. But if 
you think about this a little more, you can see that they are related: if 
$\A$ can tell the difference between two secrets based on a single share, 
it means the share gave away some information about the secret it came from.

\bigskip
Let's work through some examples. For the remainder of this section, you can 
assume that all schemes are secret sharing schemes that meet Definition~\ref{def:ss}.

\begin{example}
    Here's a secret sharing scheme for sharing integers 1 to 99 (i.e., 
    $\D = \{1, \ldots, 99\}$) among two parties:
    \begin{pchstack}[center]
    \procedure{$\share(s)$}{%
        s_1 = \text{the tens digit of } s\\
        s_2 = \text{the ones digit of } s\\
        \pcreturn (s_1, s_2)
    }
    \pchspace
    \procedure{$\rec(s_1, s_2)$}{%
        \pcreturn s_1 || s_2
    }
    \end{pchstack}
    where $||$ means concatenation (for example, $1 || 2 = 12$). Do you 
    think this scheme is private?

    Let's put ourselves in the adversary's shoes. We want to win the 
    privacy game. First, we get to pick two numbers $x_0, x_1 \in \D$ and 
    an index $i$. Let's say $x_0 = 18$ and $x_1 = 24$. We'll let $i=2$ 
    (it doesn't actually matter what $i$ is in this case). Now the game 
    picks a random $b$ unknown to us and shares $x_b$. It'll give us the 
    second share, $s_2$. 
    
    What are the possibilities for $s_2$? If $b=0$, the game runs $\share(18)$,
    which outputs $(1,8)$. Then $s_2 = 8$. If $b=1$, on the other hand, 
    the game will run $\share(24)$ to get $(2,4)$ and $s_2 = 4$. Say 
    we get $8$. Then we'll guess $b' = 0$. If we get $4$ from the game, we'll 
    guess $b'=1$. Because of how the secret sharing scheme works (``by 
    definition''), we'll always guess correctly, and $b=b'$ with probability 
    1. This means that our advantage is $\frac{1}{2}$, which is not small!
    Therefore, this scheme is not private.
\end{example}

You might say, well, duh! This was obvious from the informal definition 
of privacy: if we're given a digit of the secret, we're clearly learning 
something about the secret! Looking at the adversary's advantage in the 
game, however, makes more sense mathematically. In the case of a much 
more complicated scheme, it might be unclear what ``learning something 
about the secret'' really means.

Another thing to notice is that the adversary's choice of $x_0$ and $x_1$
is important. A less clever adversary might choose $x_0$ and $x_1$ with 
the same tens place and then ask for the $i=1$ share (the tens digit).
If it does this, it can't tell the shares apart, since they'll have the same
value. Our definition is formulated to get around issues like this: to 
show a scheme is secure, we need to show that there is \emph{no} adversary 
with a not-small advantage. This means we only need to think about what the 
cleverest way to break a scheme would be, and show that even that is 
impossible.

This is also a place where problems with cryptographic proofs of security 
can arise. There are examples of cryptographers publishing papers 
introducing new, ``secure'' schemes that end up being broken. And yet 
these schemes had proofs of security! Especially for complicated schemes, 
it sometimes turns out that the authors didn't think of the most clever 
adversary to attack the scheme. In other cases, they might assume that 
something is hard to do, so even if a clever adversary realizes that this 
is the best way to break the scheme it won't be able to pull off this 
kind of attack --- but then someone discovers a way to solve that 
hard problem. (For example, RSA cryptography, introduced in 1977\cite{rsa},
is based on the difficulty of factoring integers. But in 1994, Peter Shor 
discovered a fast way to factor integers\cite{shor} --- if you have a 
quantum computer. If you're interested in how RSA cryptography works, 
you can take a look at the RSA cryptography packet!)

\begin{example}
    % \hyperref[ex:floor]{\thesection.\ref*{ex:floor3}}
    We already saw that the scheme in Example~\exref{ex:floor3} is not correct.
    But is it private?

    You might think so at first, since the number 3 (the first share) is
    unrelated to the secret, and the second share isn't giving away the 
    secret if the adversary doesn't know that it's defined as the secret 
    divided by 3 (rounded down).
    But in reality, we can't assume that the adversary doesn't know the 
    inner workings of the \share~algorithm: we need to assume the worst.
    
    This means that $\A$ can consistently win the game if it sets $i = 2$.
    It first picks two values $x_0, x_1$ that are multiples of 3.
    When it gets $s_2$ from the game, it multiplies it by 3 and compares that 
    value to $x_0$ and $x_1$. If it equals $x_0$, it outputs $b' = 0$; 
    otherwise, it outputs $b' = 1$. By the definition of the scheme, $\A$ 
    wins with probability 1, so its advantage is $\frac{1}{2}$, and the 
    scheme is not private.
\end{example}

\begin{bonus}
    Is the following scheme private?
    \begin{pchstack}[center]
    \procedure{$\share(s)$}{%
        \pcif s \text{ is even} \pcthen\\
        \t s_1 \leftarrow \{1, \ldots, 2^{\lambda}/2\}\\
        \pcelse\\
        \t s_1 \leftarrow \{1, \ldots, 2^\lambda\}\\
        \pcreturn (s_1, s-s_2)
    }
    \pchspace
    \procedure{$\rec(s_1, s_2)$}{%
        \pcreturn s_1+s_2
    }
    \end{pchstack}
\end{bonus}

\subsection{Proving Security*}\label{sec:proof}

So far, we've only seen secret sharing schemes that were not private.
Let's work through a case in which we \emph{can} prove privacy, using the additive secret sharing 
scheme as an example.

\begin{theorem}
    The additive secret sharing scheme defined in Figure~\ref{fig:additiveSS}
    is private.
\end{theorem}
\begin{proof}
    Let $x_0$ and $x_1$ be any two integers. Define $s_{0,i}$ as
    the $i$th share output by $\share(x_0)$ and $s_{1,i}$ as the 
    $i$th share output by $\share(x_1)$.

    If $i=1$, then $s_{0,1}$ and $s_{1,1}$ are distributed uniformly 
    at random by the definition of \share~(the first share is 
    a uniformly random value). Thus, they are indistinguishable,
    and any adversary $\A$ can only randomly guess the value of $b$.
    So
    \[
        \Pr[b=b' \mid i=1] = \frac{1}{2}.
    \]

    If $i=2$, then $s_{0,2} = s-s_{0,1}$ and $s_{1,2} = s-s_{1,1}$.
    But as we said before, $s_{0,1}$ and $s_{1,1}$ are uniform,
    which implies that $s_{0,2}$ and $s_{1,2}$ are also uniform.
    Therefore they are indistinguishable, and 
    \[
        \Pr[b=b' \mid i=2] = \frac{1}{2}.
    \]

    Putting the two cases together, we see that for all possible 
    values of $x_0, x_1, i$ (and thus for all adversaries possible
    adversaries $\A$),
    \begin{align*}
        \Pr[b=b']
        =& \Pr[b=b' \mid i=1]\cdot\Pr[i=1] + \\
        & \Pr[b=b' \mid i=2]\cdot\Pr[i=2]\\
        =& \frac{1}{2} \big(\Pr[i=1]+\Pr[i=2]\big)\\
        =& \frac{1}{2}(1)\\
        =& \frac{1}{2}
    \end{align*}

    From the definition of the game, 
    \[
        \Pr[\text{SS-priv}_{\A,\ss}=1] = \Pr[b=b'],
    \]
    so the advantage for any adversary $\A$ is
    \[
        \Pr[\text{SS-priv}_{\A,\ss}=1] - \frac{1}{2} = 0
    \]
    which is clearly close to 0! So the additive secret sharing 
    scheme is private.
\end{proof}
