\section{Probability and Randomness}

\begin{answer}
    \renewcommand{\labelenumi}{(\alph{enumi})} 
    \begin{enumerate}
        \item Sample space: 
        \[
            \{A\diamondsuit, 2\diamondsuit, 3\diamondsuit, \ldots, K\diamondsuit, 
            A\heartsuit,\ldots,K\heartsuit,
            A\spadesuit,\ldots,K\spadesuit,
            A\clubsuit,\ldots,K\clubsuit\}  
        \]
        Probability distribution: 
        \[
            \left\{\frac{1}{52}, \ldots, \frac{1}{52}\right\} 
        \]
        (where $\frac{1}{52}$ is repeated 52 times).
        \item Sample space:
        \[
            \{\text{registered}, \text{not registered}\}
        \]
        Probability distribution:
        \[
            \{0.95, 0.05\}
        \]
    \end{enumerate}
\end{answer}

\begin{answer}
    \renewcommand{\labelenumi}{(\alph{enumi})} 
    \begin{enumerate}
        \item \textbf{Uniform.} The deck is shuffled, so a card is equally likely 
        to be in any one place in the deck.
        \item \textbf{Not uniform.} You are more likely to draw a lower card if 
        you draw from the beginning of the deck instead of the end.
        \item \textbf{Not uniform.} Though it fluctuates, weather generally follows 
        large-scale patterns that depend on the season, latitude and longitude,
        the air masses present, and so on. This is called climate. The reason 
        meteoroligists can predict the weather is precisely because it is 
        not a uniformly random variable.
        \item \textbf{Uniform.} Every number is equally likely to be rolled.
        \item \textbf{Not uniform.} In fact, some birthdays are more likely to occur 
        than others. Summer birthdays are slightly more common, for instance.\footnotemark
        \footnotetext{You can see a visualization of the probability of each 
        birthday here: \url{http://thedailyviz.com/2016/09/17/how-common-is-your-birthday-dailyviz/}.}
    \end{enumerate}
\end{answer}

\begin{answer}
    $d \sample \{1, 2, 3, 4, 5, 6\}$. (You could have chosen any name for the variable in the place of $d$.)
\end{answer}

\begin{answer}
    To find $\Pr[B \mid A]$, we simply switch the places of $A$ and $B$ 
    in the formula:
    \[
        \Pr[B \mid A] = \frac{\Pr[A \text{ and } B]}{\Pr[A]}
    \]
    \renewcommand{\labelenumi}{(\alph{enumi})} 
    \begin{enumerate}
        \item 0; $B$ takes up none of $A$'s space.
        \item $\frac{1}{16} \div \frac{1}{8} = \frac{1}{2}$; $B$ overlaps 
        with half of $A$.
        \item $\frac{1}{8} \div \frac{1}{8} = 1$; $B$ fully encompasses $A$, so if our 
        outcome is in $A$, it is certainly in $B$.
    \end{enumerate}
\end{answer}

\begin{answer}
    Let $c_1$ and $c_2$ be random variables representing the first and second
    coin toss, respectively. Since the coin tosses are independent, 
    \begin{align*}
        & \Pr[c_1=H \text{ and } c_2=H]\\
        =& \Pr[c_1=H] \cdot \Pr[c_2=H]\\
        =& \frac{1}{2} \cdot \frac{1}{2} = \frac{1}{4}
    \end{align*}
\end{answer}

\begin{answer}
    There are two ways of getting one heads outcome: (1) the first toss is heads 
    ($c_1=H$), and we don't do another coin toss, or (2) the first toss comes up tails,
    and the second one comes up heads ($c_1=T$ and $c_2=H$). 
    
    \begin{align*}
        & \Pr[c_1=H] + \Pr[c_1=T \text{ and } c_2=H]\\
        =& \Pr[c_1=H] + \Pr[c_2=H] \cdot \Pr[c_1=T] 
        \tag{coin tosses are independent events}\\
        =& \frac{1}{2} + \frac{1}{2}\cdot\frac{1}{2}\\
        =& \frac{1}{2} + \frac{1}{4}\\
        =& \frac{3}{4}
    \end{align*}
\end{answer}

\begin{answer}
    \renewcommand{\labelenumi}{(\alph{enumi})} 
    \begin{enumerate}
        \item Rolling an even number means rolling 2, 4, \textbf{or} 6, each 
        of which happen with probability $\frac{1}{6}$, so the probability of 
        rolling and even number is $\frac{1}{6}+\frac{1}{6}+\frac{1}{6}=\frac{1}{2}$.

        Another way to see this is that our roll has to land in the set 
        $\{2,4,6\}$ and not in $\{1,3,5\}$. Because even number happens 
        with equal probability and the sets are of equal size, the dice 
        roll lands in each of the two sets with equal probability. So landing 
        in one set happens with probability $\frac{1}{2}$.
        \item This happens if the first roll is a 1 \textbf{and} the second roll 
        is a 2. Each of those occurs with probability $\frac{1}{6}$, so the 
        answer is $\frac{1}{6}\cdot \frac{1}{6} = \frac{1}{6\cdot6}=\frac{1}{36}$.
        \item This can be broken down into cases based on the first roll:
        \begin{align*}
            &\Pr[\text{roll} > 1 \mid \text{first roll is }1]\cdot\Pr[\text{first roll is }1]\\
            &+\Pr[\text{roll} > 2 \mid \text{first roll is }2]\cdot\Pr[\text{first roll is }2]\\
            &+\Pr[\text{roll} > 3 \mid \text{first roll is }3]\cdot\Pr[\text{first roll is }3]\\
            &+\Pr[\text{roll} > 4 \mid \text{first roll is }4]\cdot\Pr[\text{first roll is }4]\\
            &+\Pr[\text{roll} > 5 \mid \text{first roll is }5]\cdot\Pr[\text{first roll is }5]\\
            &+\Pr[\text{roll} > 6 \mid \text{first roll is }6]\cdot\Pr[\text{first roll is }6]\\
        \end{align*}
        \begin{align*}
            =&\Pr[\text{roll} > 1]\cdot \frac{1}{6}
            +\Pr[\text{roll} > 2]\cdot \frac{1}{6}\\
            &+\Pr[\text{roll} > 3]\cdot \frac{1}{6}
            +\Pr[\text{roll} > 4]\cdot \frac{1}{6}\\
            &+\Pr[\text{roll} > 5\cdot] \frac{1}{6}
            +\Pr[\text{roll} > 6]\cdot \frac{1}{6}\\
            =& \frac{1}{6} \left(
                \frac{5}{6}+\frac{4}{6}+\frac{3}{6}+\frac{2}{6}+\frac{1}{6}+0
            \right)\\
            =& \frac{1}{6} \left( \frac{15}{6}\right)\\
            =& \frac{1}{6} \cdot \frac{5}{2} = \frac{5}{12}
        \end{align*}
        \item Either I roll a 5 or 6 on my first roll, or I roll a 1 and my second roll is 
        a 5 or 6:
        \begin{align*}
            &\Pr[\text{first roll is }5]
            +\Pr[\text{first roll is }6]\\
            &+\left(\Pr[\text{second roll is }5 \mid \text{first roll is }1]+
            \Pr[\text{second roll is }6 \mid \text{first roll is }1]\right)\\
            &\cdot\Pr[\text{first roll is }1]\\
            =&\frac{1}{6}+\frac{1}{6}+\left(\frac{1}{6}+\frac{1}{6}\right)\frac{1}{6}\\
            =&\frac{1}{3}+\left(\frac{1}{3}\right)\frac{1}{6}\\
            =&\frac{1}{3}+\frac{1}{18}\\
            =&\frac{7}{18}
        \end{align*}
    \end{enumerate}
\end{answer}

\begin{answer}
    \renewcommand{\labelenumi}{(\alph{enumi})}
    \begin{enumerate}
        \item \textbf{indistinguishable:} 
        both $\left\{\frac{1}{2},\frac{1}{2}\right\}$
        \item \textbf{not indistinguishable:}
        the uniform distribution with all probabilities equal to 
        $\frac{1}{52}$ is not the same as the uniform distribution 
        with all probabilities $\frac{1}{6}\left(\frac{1}{2}\right)^3 
        = \frac{1}{48}$.
        \item \textbf{indistinguishable:}
        $\left\{\frac{1}{4},3\left(\frac{1}{4}\right)\right\} 
        = \left\{\frac{13}{52},\frac{39}{52}\right\} 
        = \left\{\frac{1}{4},\frac{3}{4}\right\}$
    \end{enumerate}
\end{answer}
