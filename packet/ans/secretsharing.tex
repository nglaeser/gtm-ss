\section{Secret Sharing}

\subsection{A simple secret sharing}
\begin{sampleA}
    Note that for any secret $s$ there are many possible answers based on the 
    randomness $s_1$ that's used. 
    
    An example application of \share~with $s=42$ follows. First, $s_1$ is 
    chosen at random. Say $s_1=18$. Then \share~outputs $(18,42-18 \mod 1024)
    =(18,24 \mod 1024) = \textbf{(18,24)}$.

    $s_1$ could also be larger than $s$, e.g. $s_1 = 321$. In this case, 
    \share~outputs $(321,42-321 \mod 1024)=(321,-279 \mod 1024) = (321,
    1024-279) = \textbf{(321,745)}$.
\end{sampleA}

\begin{answer}
    Reconstruction simply adds the shares together (reducing modulo 1024):
    \renewcommand{\labelenumi}{(\alph{enumi})} 
    \begin{enumerate}
        \item $2+6 \mod 1024 =8 \mod 1024 = \textbf{8}$
        \item $4+1 \mod 1024 =5 \mod 1024 = \textbf{5}$
        \item $10+2 \mod 1024=12 \mod 1024 = \textbf{12}$
        \item $115+921 \mod 1024=1036 \mod 1024=1036-1024=\textbf{12}$
        \item $559+480 \mod 1024=1039 \mod 1024=1039-1024=\textbf{15}$
    \end{enumerate}
\end{answer}

\begin{answer} Additive secret sharing with 3 shares:
    \begin{pchstack}[center]
    \fbox{%
    \procedure{$\share(s)$}{%
        s_1, s_2 \sample \{0, \ldots, 2^\lambda-1\} \\
        s_3 = s - (s_1+s_2) \mod 2^\lambda\\
        \textsf{return}~ (s_1, s_2, s_3)
    }
    \pchspace
    \procedure{$\rec(s_1, s_2, s_3)$}{%
        \textsf{return}~ s_1 + s_2 + s_3 \mod 2^\lambda
    }
    }
    \end{pchstack}

    In fact, the additive secret sharing scheme can be adapted to share 
    the secret $s$ into any natural number $n$ of shares:
    \begin{pchstack}[center]
    \fbox{%
    \procedure{$\share(s)$}{%
        s_1, \ldots, s_{n-1} \sample \{1, \ldots, 2^\lambda\} \\
        s_n = s - (s_1+\ldots+s_{n-1}) \mod 2^\lambda\\
        \textsf{return}~ (s_1, \ldots, s_n)
    }
    \pchspace
    \procedure{$\rec(s_1, \ldots, s_n)$}{%
        \textsf{return}~ s_1 + \ldots + s_n \mod 2^\lambda
    }
    }
    \end{pchstack}

    (Where if $\rec$ is run on the incorrect number of shares---anything except 
    $n$---the algorithm returns $\perp$.)
\end{answer}

\subsection{Formal Definitions*}
\begin{sampleA}
    The person playing the role of the adversary should only be able to win about half the time. This is 
    because $s_i$ looks random to the adversary. The key part of the scheme 
    is that $s_1$ is chosen uniformly at random, thereby making both $s_1$ 
    and $s_2$ uniformly distributed and independent of $s$.
\end{sampleA}

\begin{answer}
    No. There is an adversary $\A$ whose advantage in the privacy game 
    is not small. 
    
    $\A$ works as follows: it chooses values $x_0, x_1$ such that $x_0$ 
    is even and $x_1$ is odd (the opposite works too) and lets $i=1$. 
    When the game sends it the share $s_1$, $\A$ checks if $s_1 < 2^\lambda/2$. 
    If so, it outputs $b'=0$; otherwise, it outputs $b'=1$.

    $\A$ wins the game with probability $3/4$:

    \begin{align*}
        \Pr[\text{SS-priv}_{\A,\ss}(t,n) = 1] \\
        =& \Pr[b'=b \mid b=0] + \Pr[b'=b \mid b=1]\\
        =& \frac{1}{2}\left(1\right) + \frac{1}{2}\left(\frac{1}{2}\right)\\
        =& \frac{1}{2} + \frac{1}{4} = \frac{3}{4}
    \end{align*}
    \smallskip

    So $\A$'s advantage is $\frac{3}{4}-\frac{1}{2} = \frac{1}{4}$, which 
    is not small.
\end{answer}

% \subsection{Proving Security*}
