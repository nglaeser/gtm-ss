\documentclass[12 pt]{article}
\hbadness=10


\usepackage{amsmath, amssymb, amsfonts, setspace, stmaryrd, amsthm, graphicx, tikz}

\usepackage[utf8]{inputenc}
\usepackage[english]{babel}
\usetikzlibrary{positioning}% To get more advanced positioning options
\usetikzlibrary{arrows}% To get more arrow heads
\usepackage{hyperref} % Allows to make references and links clickable

\usepackage[margin=1.5in]{geometry}
\newtheorem{theorem}{Theorem}
%%align* environment
\newcommand{\eq}[1]{\begin{align*}#1\end{align*}}

%Math commands
\DeclareMathOperator{\R}{\mathbb{R}}
\DeclareMathOperator{\C}{\mathbb{C}}
\DeclareMathOperator{\Z}{\mathbb{Z}}
\DeclareMathOperator{\N}{\mathbb{N}}
\DeclareMathOperator{\Q}{\mathbb{Q}}

\renewcommand\abstractname{Introduction}

\newcounter{exercise}[section]
\newenvironment{exercise}{\refstepcounter{exercise}\par\bigskip \begin{quotation}{}{\leftmargin .25in\rightmargin .25in}
	\noindent \textbf{Exercise~\thesection.\theexercise }  \rmfamily}{\end{quotation}\par\bigskip}
	

\newenvironment{bonus}{\refstepcounter{exercise}\par\bigskip \begin{quotation}{}{\leftmargin .25in\rightmargin .25in}
	\noindent \textbf{*Bonus Exercise*~\thesection.\theexercise }  \rmfamily}{\end{quotation}\par\bigskip}
	 


\title{Cryptographic Secret Sharing}
\author{Girls Talk Math}
\date{}

\begin{document}
\maketitle
\vskip 1in
\begin{center} \textbf{Introduction} \end{center}

\indent In this problem set, you will learn about ...
    
    One last note about reading mathematical texts: it is very normal when reading math to read a passage or even a single sentence several times before understanding it properly. Also, never trust the author! Check every claim and calculation (time permitting). Take your time and never give up. Let's talk math!
	
\newpage

\tableofcontents

%%%%%%%%%%%%%%%%%%%%%%
%%% Use \section{}, \subsection{}, etc. for different parts of the problem set. Start a newpage for a new section
%%%%%%%%%%%%%%%%%%%%%%

%%%%%%%%%%%%%%%%%%%%%%%%%%%%%%%%%
%%%
%%% Embed exercises in section as appropriate 
%%% 
%%%%%%%%%%%%%%%%%%%%%%%%%%%%%%%%%%%

\newpage


\section{Probability and Randomness}

\subsection{Introduction}

\subsection{Randomness in Cryptography}

\subsection{Sharing Secrets}
\emph{Secret sharing} is a way to...

\section{A simple secret sharing}

\subsection{Binary Arithmetic}

The exclusive-OR (XOR) operation is denoted by the symbol $\oplus$ and defined by the following truth table:

\begin{center}
\begin{tabular}{c|c|c}
    $a$ & $b$ & $a \oplus b$\\\hline
    0 & 0 & 0\\
    0 & 1 & 1\\
    1 & 0 & 1\\
    1 & 1 & 0
\end{tabular}
\end{center}

\subsection{Sharing Secrets using XOR}

\newpage
\section{Shamir's Secret Sharing}
\subsection{Polynomials}
\subsubsection{Uniqueness}

\begin{exercise}
\end{exercise}

\subsection{Sharing Secrets Using Polynomials}

\end{document}
